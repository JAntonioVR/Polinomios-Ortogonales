% ─── ABSTRACT ─────────────────────────────────────────────────────────────────

Approximation in mathematics has always been one of main alternatives to face the difficulty of obtaining exact solutions to some problems. In this context we can find the orthogonal polynomials, which can be used for, inter alia, approximate functions or calculate integrals and transforms.

On the other hand, the interest and the need to predict future events has caused the origin of many probabilistic models, among which we highlight the Markov chains. In particular, Markov chains and processes are state models, i.e., they try to predict the future situation of a system starting from a very limited information.

It looks like we have been talking about two very different mathematical areas so far, because both disciplines has distinct origin and aims. Nevertheless, there exist such a close relation between orthogonal polynomials and Markov chains. This relation lies in the birth-and-death processes, which are a particular case of Markov chains about which the mathematicians Samuel Karlin (1994--2007) and James McGregor (1926--1991) wrote.

In this final master's project we will start studying orthogonal polynomials and Markov chains independently in order to finish explaining the relation between them. Thus, this document has been structured in four chapters: 

\begin{enumerate}
    \item \textbf{Introduction to Orthogonal Polynomials}: The main concepts and results about orthogonal polynomials theory are explained. The base notation is also established and some clarifying examples are exposed.
    \item \textbf{Classic Orthogonal Polynomials}: The known as `classic' orthogonal polynomials are presented. They are particular families of polynomials which share some interesting properties that characterize them.
    \item \textbf{Markov Chains}: Once the orthogonal polynomials theory has been introduced, Markov chains and their behavior are studied in a fully probabilistic way, showing several examples.
    \item \textbf{Orthogonal Polynomials and Birth-and-Death Processes}: Having the two preliminary theories explained, birth-and-death processes are defined and the relation between these probabilistic models and orthogonal polynomials are deduced.
\end{enumerate}

In addition, throughout this document we show some examples for which mathematical software tools were used to illustrate the theoretical concepts in a practical way. Through the use of \texttt{SageMath} and \texttt{R}, some symbolic and numeric calculations, graphic renders, temporary measures and comparisons have been made. The source code of this software can be found in the project's github repository: \url{https://github.com/JAntonioVR/Polinomios-Ortogonales/tree/main/software}.



\section*{Keywords}

Orthogonal Polynomials, Classic Orthogonal Polynomials, Markov Chains, Birth-and-Death Processes, Representation Formula, Mathematical Analysis, Probabilistic Analysis, Stochastic Processes, Mathematical Software.
% ─── RESUMEN ──────────────────────────────────────────────────────────────────

La aproximación en matemáticas siempre ha sido uno de las principales alternativas ante la dificultad de encontrar soluciones exactas a algunos problemas. En este marco se sitúan los polinomios ortogonales, útiles para, entre otros ámbitos, aproximar funciones o calcular integrales y transformadas. 

Por otro lado, el interés y la necesidad de predecir sucesos futuros ha provocado que se desarrollen múltiples modelos probabilísticos, entre los que hoy destacamos las cadenas de Markov. En particular, las cadenas y procesos de Markov son modelos de estados, es decir, tratan de predecir la situación en la que se encontrará en un futuro un sistema a partir de muy poca información. 

Podría parecer que hasta el momento se ha hablado de dos áreas muy distintas de las matemáticas, pues tanto su naturaleza como sus objetivos son notablemente diferentes. Sin embargo, existe una relación muy estrecha entre los polinomios ortogonales y las cadenas de Markov. Esta relación reside en los procesos de nacimiento y muerte, que son un caso particular de cadenas de Markov sobre las cuales estudiaron los matemáticos Samuel Karlin (1924--2007) y James McGregor (1926--1991).

En este trabajo estudiamos en principio de forma independiente los polinomios ortogonales y las cadenas de Markov para finalmente exponer la relación entre ambas. Así, la memoria de este trabajo se ha estructurado en cuatro capítulos:

\begin{enumerate}
    \item \textbf{Introducción a los Polinomios Ortogonales}: Se explican los conceptos y resultados más elementales de la teoría de polinomios ortogonales, se establece la notación que se usará a lo largo del trabajo y se exponen algunos ejemplos clarificadores.
    \item \textbf{Polinomios Ortogonales Clásicos}: Se presentan los polinomios ortogonales conocidos como clásicos, que son familias particulares con varias propiedades exclusivas que las caracterizan.
    \item \textbf{Cadenas de Markov}: Ya introducida la teoría de polinomios ortogonales, se estudian desde un punto de vista puramente probabilístico las cadenas de Markov y su comportamiento, exponiendo diversos ejemplos.
    \item \textbf{Polinomios Ortogonales y Procesos de Nacimiento y Muerte}: Una vez explicadas las dos disciplinas preliminares, se definen los procesos de nacimiento y muerte y se deduce la relación entre estos modelos probabilísticos y los polinomios ortogonales.
\end{enumerate}

Además, a lo largo del documento se muestran ejemplos para los cuales se utilizaron herramientas de software matemático con el fin de ilustrar de manera práctica los conceptos teóricos. Mediante el uso de \texttt{SageMath} y de \texttt{R}, se llevaron a cabo cálculos simbólicos y numéricos, visualizaciones, medidas temporales y comparaciones que acercan al lector a casos concretos de teorías abstractas. El código fuente de este software puede ser consultado en el repositorio de github del trabajo: \url{https://github.com/JAntonioVR/Polinomios-Ortogonales/tree/main/software}.


\section*{Palabras clave}

Polinomios Ortogonales, Polinomios Ortogonales Clásicos, Cadenas de Markov, Procesos de Nacimiento y Muerte, Fórmula de Representación, Análisis Matemático, Análisis Probabilístico, Procesos Estocásticos, Software Matemático.
%
% ──────────────────────────────────────────────────────────────── I ──────────
%   :::::: I N T R O D U C C I O N : :  :   :    :     :        :          :
% ──────────────────────────────────────────────────────────────────────────
%

Tal y como los conocemos hoy, los polinomios tienen una antigua historia, que se remonta a las matemáticas clásicas de la antigua Grecia y el mundo islámico medieval, precursores en el estudio de curvas cónicas (que como sabemos pueden ser modeladas a partir de polinomios) y en la resolución de ecuaciones polinómicas.

En el siglo XVII, el matemático y físico holandés \textsc{Christian Huygens} (1629--1695) fue uno de los primeros en estudiar lo que hoy conocemos como \textit{polinomios ortogonales}. En su obra ``De Circuli Magnitudine Inventa'' (1654), introdujo los polinomios de Hermite, que surgieron en el estudio de la vibración de cuerdas y el comportamiento de los fluidos. Un siglo después, fue el suizo \textsc{Leonhard Euler} (1707--1783) quien hizo importantes contribuciones al estudio de los polinomios ortogonales, trabajando con familias de  polinomios como las de Legendre y Laguerre, entre otros, en problemas físicos y matemáticos. Ya en el siglo XIX, el matemático alemán \textsc{Gustav Lejeune Dirichlet} (1805--1859) desempeñó un papel fundamental en el estudio de los polinomios ortogonales. Estableció criterios rigurosos para la ortogonalidad y la normalización de los polinomios y los aplicó en el campo del análisis de Fourier.

A partir de los trabajos de Huygens, Euler, Dirichlet y otros matemáticos, el estudio de los polinomios ortogonales se fue expandiendo y se establecieron conexiones con diversas áreas de las matemáticas, como el análisis funcional, la teoría de aproximación, la teoría de las ecuaciones diferenciales y la teoría de la probabilidad, siendo esta última la que ocupará uno de los principales objetivos de este trabajo.

La teoría de polinomios ortogonales tiene, como hemos mencionado, aplicaciones y conexiones con distintas ramas de las matemáticas. En concreto, nos centraremos en la teoría probabilística de los procesos y cadenas de Markov. En 1906, \textsc{Andréi Markov} (1856--1922) introdujo por primera vez el concepto de \textit{cadenas de Markov} como un modelo probabilístico para describir secuencias de eventos o procesos que exhiben una propiedad importante conocida como "la propiedad de Markov". Esta propiedad establece que la probabilidad de que ocurra un evento en el futuro depende únicamente del estado presente y no de los estados pasados. Dicho así, pueden observarse diferencias muy notorias entre ambas disciplinas, pues una es fundamentalmente analítica y otra probabilística, una tiene aplicaciones en el campo de las ecuaciones diferenciales y teoría de la aproximación, otra trata de establecer modelos probabilísticos con el objetivo de predecir el futuro.

A pesar de sus disparidades, a lo largo de este trabajo buscaremos qué posible relación existe entre estas dos áreas. Esta conexión se encuentra en los conocidos \textit{procesos de nacimiento y muerte}, que en realidad son un caso particular de cadena de Markov infinita. Fueron los matemáticos \textsc{Samuel Karlin} (1924--2007) y \textsc{James McGregor} (1921--1988) quienes estudiaron la forma de aplicar polinomios ortogonales en el cálculo de probabilidades, utilizando la que llamaremos fórmula de representación de Karlin y McGregor.

En el proceso de búsqueda de esta singular conexión entre polinomios ortogonales y cadenas de Markov trataremos primero de sentar bases para su correcta comprensión, utilizando herramientas de álgebra lineal, análisis matemático básico, análisis funcional, teoría de la probabilidad, modelos de estados, entre otras. Por ello, recomendamos que el lector cuente con ciertas nociones de las materias que describimos a continuación:
\begin{itemize}
    \item Álgebra lineal básica: El conjunto de todos los polinomios es, de por sí, un espacio vectorial, por lo que a lo largo de todo el documento se hará uso extensivo de las propiedades de linealidad. Además, la forma de establecer una ortogonalidad, en un primer instante, es mediante el uso de un producto escalar. También es conveniente que el lector recuerde las propiedades fundamentales de las matrices.
    \item Análisis matemático básico: En realidad, los polinomios no dejan de ser funciones, que de hecho son continuas, infinitamente derivables, integrables, definidas en toda la recta real\dots En varias ocasiones se utilizan integrales para definir productos escalares y funcionales lineales.
    \item Teoría de la medida: Aunque no con demasiada profundidad, es aconsejable conocer algunas nociones sobre teoría de la medida. En concreto, medidas inducidas por funciones positivas, funciones peso, etc.
    \item Análisis funcional: Otra forma de establecer ortogonalidad es utilizando funcionales lineales, por lo que se recomienda conocer su definición y propiedades más básicas. También se utilizan conceptos como espacio de Hilbert, operadores lineales y continuos, norma de operadores y teoría espectral.
    \item Probabilidad: Aunque se explica desde las bases toda la teoría de cadenas de Markov y sus comportamientos, es conveniente partir de cierta base de teoría de la probabilidad, conociendo términos como espacio muestral, variable aleatoria o medida de probabilidad.
    \item Programación en \texttt{SageMath} y \texttt{R}: A modo aclaratorio, hemos utilizado software de cálculo simbólico y numérico para exponer algunos ejemplos. Se ha utilizado \texttt{SageMath} para aquellos ejemplos que requerían cálculos simbólicos complejos, y \texttt{R}, que es un conocido software estadístico, para algunos ejemplos de cadenas de Markov.
    \item \cb{REVIEW Me dejo algo?}
\end{itemize}

Utilizando estas bases, la memoria se estructura en cuatro capítulos: En el primer capítulo se estudian los polinomios ortogonales desde un punto de vista general, exponiendo sus principales definiciones y resultados. Seguidamente, en el segundo capítulo abandonamos esta generalidad para centrarnos en las familias de polinomios ortogonales más útiles y conocidas: los polinomios ortogonales clásicos. A continuación, en el tercer capítulo, nos apartamos completamente la teoría de polinomios ortogonales para centrarnos en la teoría probabilística de las cadenas de Markov, definiendo estos modelos e introduciendo resultados que facilitan el cálculo de probabilidades y caracterizan el comportamiento a largo plazo de la cadena. Finalmente, en el cuarto y último capítulo, una vez conocidos los polinomios ortogonales y las cadenas de Markov, abordaremos el objetivo fundamental del trabajo: la relación entre estas dos disciplinas. Definimos los procesos de nacimiento y muerte, que jugarán un papel esencial, y posteriormente probaremos la fórmula que relaciona probabilidades, integrales y polinomios ortogonales: la fórmula de representación de Karlin-McGregor, enfocándonos con mayor detalle en su versión discreta.

Cabe destacar el ya mencionado uso de software matemático como \href{https://www.sagemath.org/}{\texttt{SageMath}} y \href{https://www.r-project.org/}{\texttt{R}} en el desarrollo de algunos ejemplos a lo largo del trabajo. El código fuente de todos los códigos utilizados se puede encontrar en el repositorio de Github del proyecto, concretamente en la URL \url{https://github.com/JAntonioVR/Polinomios-Ortogonales/tree/main/software}. En particular, desde el propio navegador es posible abrir los archivos con extensión \texttt{.ipynb} para visualizar los códigos utilizados para ciertos cálculos y generación de gráficas, explicados paso a paso. Por su parte, recomendamos descargar y abrir en un navegador el archivo \href{https://github.com/JAntonioVR/Polinomios-Ortogonales/blob/main/software/ejemplo-migraciones.html}{\texttt{ejemplo-migraciones.html}} para observar el código utilizado en los ejemplos \ref{ej:cv} y \ref{ej:cv2}. Para usuarios más experimentados, es posible descargar todos estos ficheros y ejecutarlos utilizando un `notebook' de \href{https://jupyter.org/}{\texttt{Jupyter}} en el caso de los archivos de \texttt{SageMath} o el software \href{https://posit.co/download/rstudio-desktop/}{\texttt{RStudio}} para el archivo de \texttt{R}.

Finalmente, comentaremos los recursos bibliográficos en los que, de manera fundamental, nos hemos basado para el desarrollo de esta memoria. Para la introducción a los polinomios ortogonales nos hemos basado en los capítulos introductorios de \cite{renato} y \cite{chihara}, que también fueron útiles en la explicación de los polinomios ortogonales clásicos junto con el artículo  \cite{Marcellán1994}. En la exposición de las cadenas de Markov hemos utilizado principalmente \cite{kulkarni-2012} y \cite{Ross}. Y por último, en el capítulo último nos hemos basado en algunos de los artículos de Karlin y McGregor \cite{differential-equations}, \cite{Linear-Growth}, \cite{random-walks} y su adaptación en \cite{Manuel}. Los mencionados componen la bibliografía básica por capítulos, aunque cabe destacar las referencias \cite{abramowitz-stegun} y \cite{szego} como recursos utilizados en varios capítulos gracias a su completitud.

Sin mucho más que añadir, el humilde autor de este trabajo desea una experiencia agradable durante su lectura y el disfrute de cada una de las secciones.


%%%%%%%%%%%%%%%%%%%%%%%%%%%%%%%%%%%%%%%%%%%%%%%%%%%%%%%%%%%
%% Capítulo 2: Polinomios Ortoganles Clásicos            %%
%%%%%%%%%%%%%%%%%%%%%%%%%%%%%%%%%%%%%%%%%%%%%%%%%%%%%%%%%%%


En el capítulo \ref{chap:introduccionPO} hemos mostrado una amplia introducción a la ortogonalidad y presentado ejemplos concretos de polinomios ortogonales. En este capítulo presentaremos las familias concretas de polinomios ortogonales más importantes: los \textit{polinomios ortogonales clásicos}. Estos son los polinomios de Hermite, Laguerre, Jacobi y Bessel. Estas familias presentan la peculiaridad de ser las únicas que verifican ciertas propiedades, entre las que destaca la \textit{ecuación diferencial de Pearson}. Estos polinomios, entre otras disciplinas pueden ser encontrados en problemas de Sturm-Liouville cuando se utilizan ecuaciones diferenciales hipergeométricas.

\begin{figure}[h]
    \centering
    \begin{tabular}{ccc}
        \includegraphics[width=5cm]{img/C2/hermite.png} & 
        \includegraphics[width=5cm]{img/C2/laguerre.png} &
        \includegraphics[width=5cm]{img/C2/jacobi.png} \\
        (a) Hermite $H_n$ & (b) Laguerre $L^{(0)}_n$ & (c) Jacobi $J^{(0,0)}_n$ 
    \end{tabular}
    \caption{Polinomios Ortogonales Clásicos}
    \label{img:graficas-clasicos}
\end{figure}

En las imágenes \ref{img:graficas-clasicos} podemos ver la representación gráfica, para parámetros concretos que definiremos próximamente, de los polinomios ortogonales de Hermite, Laguerre y Jacobi. Estas tres son las familias que mayor atención recibirán de nuestra parte al ser aquellas cuya ortogonalidad se manifiesta en intervalos reales.

\section{La ecuación de Pearson}

Para empezar, introduciremos una nueva notación para los funcionales de momentos que también se suele utilizar. Sea $\uu$ un funcional de momentos y $\{\mu_n\}$ su sucesión de momentos, entonces denotamos
\begin{equation}
    \label{eq:nueva-notacion-funcional}
    \begin{split}
        \uu:\mathbb P & \longrightarrow \R \\
        x^n & \longmapsto \left\langle \uu, x^n \right\rangle = \mu_n
    \end{split}
\end{equation}

A lo largo de este capítulo denotaremos como $\prodesc{\uu}{\pi(x)}$ a la aplicación del funcional $\uu$ a un polinomio $\pi(x)$.

Definimos a continuación dos operadores que actuán sobre los funcionales.

\begin{definicion}[Producto por un polinomio]
    Para cada polinomio $\pi$, definimos un nuevo funcional de momentos a partir de $\uu$ como
    \begin{equation}
        \begin{split}
            \pi\uu:\mathbb P & \longrightarrow \R \\
            \phi & \longmapsto \prodesc{\pi\uu}{\phi}=\prodesc{\uu}{\pi\phi}
        \end{split}
    \end{equation}
    
\end{definicion}

\begin{definicion}[Derivada]
    Definimos el funcional derivada como
    \begin{equation}
        \begin{split}
            D(\uu):\mathbb P & \longrightarrow \R \\
            \phi & \longmapsto \prodesc{D(\uu)}{\phi}=-\prodesc{\uu}{\phi'}
        \end{split}
    \end{equation}
    
\end{definicion}

A partir de estos dos operadores y esta nueva notación daremos una primera definición de lo que es una SPO clásica.

\begin{definicion}[SPO Clásica]
    La SPO $\{P_n  \}$ para el funcional $\uu$ se dice que es \textbf{clásica} (Hermite, Laguerre, Jacobi o Bessel) si existen polinomios $\sigma(x), \tau(x)$ con $\deg(\sigma)\leq 2$ y $\deg(\tau)= 1$ tales que el funcional $\uu$ verifica la ecuación diferencial
    \begin{equation}
        \label{eq:Pearson-u}
        D(\sigma \uu) = \tau \uu
    \end{equation}   
    La ecuación (\ref{eq:Pearson-u}) es conocida como la \textbf{ecuación de Pearson}.
\end{definicion}

En el caso de que la SPO sea ortogonal respecto a una función peso $\rho$, esto es, $\uu$ es de la forma
\begin{equation}
    \label{eq:funcional-peso}
    \prodesc{\uu}{\pi(x)} = \int_a^b \pi(x)\rho(x)dx,
\end{equation}
donde $(a,b)$ es cierto intervalo de la recta real donde $\rho > 0$, podemos deducir una condición equivalente.

\begin{lema}
    \label{lema:equivalencia-pearson}
    Sean $\rho(x)$ una función peso positiva en el intervalo $(a,b)$,$\uu$ el funcional definido como en (\ref{eq:funcional-peso}) y $\{P_n\}$ la SPO respecto a $\uu$. Si la función peso $\rho(x)$ es solución de la ecuación diferencial
    \begin{equation}
        \label{eq:Pearson-peso}
        [\sigma(x)\rho(x)]'=\tau(x)\rho(x)
    \end{equation}
    y además verifica las condiciones de frontera 
    \begin{equation}
        \label{eq:cond-frontera}
        \displaystyle\lim_{x\rightarrow a}\sigma(x)\rho(x)x^n = \displaystyle\lim_{x\rightarrow b}\sigma(x)\rho(x)x^n, \ \ \ n\geq 0,
    \end{equation}
    entonces la SPO $\{P_n\}$ es clásica, \textit{i.e.} $\uu$ verifica la ecuación de Pearson (\ref{eq:Pearson-u}).
\end{lema}
\begin{proof}
    Queremos comprobar la igualdad de funcionales $D(\sigma \uu)=\tau\uu$. Para ello, tengamos en cuenta que dos funcionales lineales son iguales si, y solo si actuán igual sobre una base de $\mathbb P$. Escogemos la base $\{x^n\}_{n\geq 0}$. Entonces tenemos que
    $$
    \prodesc{D(\sigma \uu)}{x^n} = -\prodesc{\sigma \uu}{n x^{n-1}}= -\prodesc{\uu}{\sigma(x)n x^{n-1}},
    $$
    aplicando (\ref{eq:funcional-peso}) e integración por partes, llegamos a
    \begin{equation*}
        \begin{split}
            \prodesc{D(\sigma \uu)}{x^n} &= -\int_a^b \sigma(x)n x^{n-1}\rho(x)dx = \left\{\begin{array}{ll}
                u = \sigma(x)\rho(x) & du=[\sigma(x)\rho(x)]'dx \\
                dv = n x^{n-1}dx & v = x^n
            \end{array}\right\} \\
            &= -\underbrace{\left[\sigma(x)\rho(x)x^n\right]_a^b}_{=0\text{ por (\ref{eq:cond-frontera})}}+ \int_a^b x^n [\sigma(x)\rho(x)]'dx\ \ \ \ \  \text{(aplicando (\ref{eq:Pearson-peso}))} \\
            &= \int_a^b x^n \tau(x)\rho(x)dx = \prodesc{\uu}{\tau(x)x^n} = \prodesc{\tau \uu}{x^n}
        \end{split}
    \end{equation*}
    
\end{proof}

\cb{REVIEW El recíproco tiene que ser cierto, pero no tengo la demostración }

\section{Deducción de las familias de polinomios ortogonales clásicos}

A partir del lema \ref{lema:equivalencia-pearson} podremos obtener las principales familias de polinomios ortogonales. Hemos impuesto que $\tau(x)=Ax+B$ sea un polinomio de grado exactamente $1$, por lo que tenemos cuatro grados de libertad a partir de las posibilidades de $\sigma(x)$, que supondremos mónico:

\begin{enumerate}
    \item \textbf{Caso I}: $\sigma(x) = 1$, $x\in\R$. En este caso obtendremos los llamados polinomios de Hermite.
    \item \textbf{Caso II}: $\sigma(x) = x-a$, $x\in[a,\infty)$. Haciendo el cambio de variable lineal $t=-(x-a)/A$ se tiene $\sigma(x) = x$ y $\tau(x)=-x+B$, $x\in[0,\infty)$. Con estos valores calcularemos los polinomios de Laguerre.
    \item \textbf{Caso III}: $\sigma(x) = (x-a)(b-x)$, $x\in[a,b]$. Con el cambio de variable $x = (b-a)/2t + (a+b)/2$ podemos escribir $\sigma(x)=1-x^2$ y $\tau=Ax+B$, $x\in[-1,1]$. Y así deduciremos los polinomios de Jacobi.
    \item \textbf{Caso IV}: $\sigma(x) = (x-a)^2$. En este último caso se obtienen los polinomios de Bessel. 
\end{enumerate}

De acuerdo al lema \ref{lema:equivalencia-pearson}, si resolvemos la ecuación diferencial (\ref{eq:Pearson-peso}) asegurándonos de que las soluciones calculadas verifiquen las condiciones de frontera (\ref{eq:cond-frontera}), habremos hayado las funciones peso $\rho(x)$ para las cuales el funcional (\ref{eq:funcional-peso}) genera una SPO clásica.Veremos seguidamente los casos I, II y III, obviando los polinomios de Bessel.

\subsection{Caso I: Polinomios de Hermite}

Supongamos que $\sigma(x)=1$, $\forall x\in \R$. El caso puede ser reducido a $\tau(x)=-2x$. Tenemos entonces la ecuación diferencial
$$
\rho'(x) = -2x \rho(x), \ \ x\in\R.
$$
Esta ecuación es de variables separadas, por lo que podemos tomar
$$
\int \dfrac{\rho'(x)}{\rho(x)}dx = \int -2x dx \Leftrightarrow \log(\rho(x)) = -x^2 + log(C) \Leftrightarrow \rho(x) = C e^{-x^2} \neq 0 \ \ \forall x \in \R,
$$
tomamos $C=1$ y llegamos a 
\begin{equation}
    \label{eq:parametros-hermite}
    \begin{array}{ccc}
        \sigma(x)=1, & \tau(x)=-2x, & \rho(x) = e^{-x^2}, \ \ \forall x \in \R.
    \end{array}
\end{equation}
Sobre las condiciones de frontera, tenemos que $\displaystyle\lim_{x\rightarrow\pm\infty} e^{-x^2}x^n = 0$ $\forall n\in\N_0$. Por tanto, la sucesión de polinomios ortogonales con respecto al funcional
\begin{equation}
    \label{eq:func-hermite}
    \prodesc{\uu}{\pi(x)} = \int_{-\infty}^\infty \pi(x)e^{-x^2}dx
\end{equation}
es clásica, y sus elementos son los \textbf{polinomios de Hermite}.

\subsection{Caso II: Polinomios de Laguerre}

Supongamos que $\sigma(x) = x$ y $\tau(x)=-x+B$ con $x\in[0,+\infty)$. La ecuación (\ref{eq:Pearson-peso}) queda entonces como
$$
(x\rho(x))'=(-x+B)\rho(x),
$$
que si derivamos el primer producto y agrupamos equivale a $x\rho'(x)=(B-x-1)\rho(x)$, que es de nuevo una ecuación de variables separadas, de manera que
$$
\int \dfrac{\rho'(x)}{\rho(x)}dx = \int \dfrac{B-x-1}{x}dx \Leftrightarrow \log(\rho(x)) = B\log(x)-x-\log(x) + \log(C)\Leftrightarrow $$ $$
 \rho(x) = Ce^{-x}x^{B-1}\neq 0 \ \ \forall x\in[0,+\infty),
$$
Tomamos $C=1$ y definimos $\alpha = B-1$, de forma que
\begin{equation}
    \label{eq:parametros-laguerre}
    \begin{array}{cccc}
        \sigma(x)=x, & \tau(x)=-x+\alpha+1, & \rho(x) = x^{\alpha} e^{-x}\ \ \forall x \in[0,+\infty), & \alpha > -1.
    \end{array}
\end{equation}
La condición $\alpha > -1$ nos asegura que los momentos del funcional serán finitos, es decir, $\mu_n =\int_0^\infty x^nx^{\alpha} e^{-x}dx <\infty$ $\forall n\geq 0$.

En este caso, las condiciones de frontera también se cumplen, pues $\displaystyle\lim_{x\rightarrow 0} x\cdot x^{\alpha}e^{-x}\cdot x^n = \displaystyle\lim_{x\rightarrow \infty} x\cdot x^{\alpha}\cdot e^{-x} x^n = 0$. Así, la SPO con respecto al funcional
\begin{equation}
    \label{eq:func-laguerre}
    \prodesc{\uu}{\pi(x)} = \int_{0}^\infty \pi(x)x^{\alpha} e^{-x}dx, \ \ \alpha > -1
\end{equation}
es clásica, y sus elementos son los \textbf{polinomios de Laguerre}.

\subsection{Caso III: Polinomios de Jacobi}

Por último, consideremos $\sigma(x) = 1-x^2$ y $\tau(x)=Ax+B$ con $x\in[-1,1]$. Aplicado a la ecuación (\ref{eq:Pearson-peso}) tenemos
$$
((1-x^2)\rho(x))'=(Ax+B)\rho(x),
$$
Derivando el primer miembro y agrupando obtenemos a $(1-x^2)\rho'(x)=(Ax+B+2x)\rho(x)$. Si dividimos y multiplicamos el segundo miembro por $\sigma(x)$ y dividimos la ecuación por $\sigma(x)\rho(x)$ obtenemos la ecuación
$$
\dfrac{\sigma(x)\rho(x)}{\sigma(x)\rho(x)} = \dfrac{Ax+B}{1-x^2}.
$$
Resolveremos esta ecuación en la que $(\sigma\rho)$ es la incógnita. Una vez resuelta podremos deducir la solución de (\ref{eq:Pearson-peso}).
$$
\int \dfrac{\sigma(x)\rho(x)}{\sigma(x)\rho(x)} dx = \int \dfrac{Ax+B}{1-x^2}dx\Leftrightarrow $$ $$\log(\sigma(x)\rho(x)) = -\dfrac{A+B}{2}\log(1-x)-\dfrac{A-B}{2}\log(1+x) + \log(C) \Leftrightarrow $$ $$
 \sigma(x)\rho(x) = C(1-x)^{-\frac{A+B}{2}}(1+x)^{-\frac{A-B}{2}}  \Leftrightarrow$$ $$\rho(x) = C(1-x)^{-\frac{A+B}{2}-1}(1+x)^{-\frac{A-B}{2}-1} \neq 0 \ \ \forall x\in[-1,1].
$$
Tomamos $C=1$ y definimos $\alpha = -\frac{A+B}{2}-1$ y $\beta=-\frac{A-B}{2}-1$, obteniendo
\begin{equation}
    \label{eq:parametros-jacobi}
    \begin{array}{c}
        \sigma(x)=1-x^2,\hspace{2cm} \tau(x)=-(\alpha+\beta+2)x+(\beta-\alpha), \\ 
        \rho(x) =(1-x)^{\alpha}(1+x)^{\beta}\ \ \forall x \in[-1,1], \ \ \alpha,\beta > -1.
    \end{array}
\end{equation}

Al igual que en el caso anterior, las condiciones $\alpha,\beta > -1$ nos aseguran que los momentos del funcional serán finitos, es decir, $\mu_n =\int_{-1}^1 x^n(1-x)^{\alpha}(1+x)^{\beta}dx <\infty$ $\forall n\geq 0$.

Las condiciones de frontera se verifican nuevamente, pues $$\displaystyle\lim_{x\rightarrow \pm 1} (1-x^2)\cdot (1-x)^{\alpha}(1+x)^{\beta}\cdot x^n = 0.$$

Así, la SPO con respecto al funcional
\begin{equation}
    \label{eq:func-jacobi}
    \prodesc{\uu}{\pi(x)} = \int_{-1}^1 \pi(x)(1-x)^{\alpha}(1+x)^{\beta}dx, \ \ \alpha,\beta > -1
\end{equation}
es clásica, y sus elementos son los \textbf{polinomios de Jacobi}.



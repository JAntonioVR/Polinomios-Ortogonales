
%%%%%%%%%%%%%%%%%%%%%%%%%%%%%%%%%%%%%%%%%%%%%%%%%%%%%%%%%%%
%% Capítulo 1: Introducción a los Polinomios Ortogonales %%
%%%%%%%%%%%%%%%%%%%%%%%%%%%%%%%%%%%%%%%%%%%%%%%%%%%%%%%%%%%

\section{Introducción}
\label{c1section:intro}

La \textit{Ortogonalidad} es una propiedad que a menudo se ha relacionado con la geometría, siendo común pensar en la analogía con la \textit{perpendicularidad}. Por ejemplo, es claro que en el plano vectorial $\R^2$ se tiene que los vectores $(1,0)$ y $(0,1)$ son ortogonales, y es que estos forman un ángulo recto, y por ello también se dice que son perpendiculares. Sin embargo, este hecho no es más que una consecuencia de la ortogonalidad, y es que en el espacio vectorial $\R^d$, y en particular en $\R^2$, dos vectores se dicen ortogonales si, al dotar a $\R^d$ de un producto escalar $\left\langle\cdot,\cdot\right\rangle:\R^d\times\R^d\longrightarrow \R$, el resultado de operar estos dos vectores es $0$.

El producto escalar mayormente utilizado en $\R^d$ es el usual, el cual, si $u=(u_1,\dots,u_d)$ y $v=(v_1,\dots,v_d)$ se define como
\begin{equation*}
    \begin{split}
        \left\langle\cdot,\cdot\right\rangle:\R^d\times\R^d &\longrightarrow \R \\
        (u,v)&\longmapsto \left\langle u, v \right\rangle=u\cdot v^t = \sum_{i=1}^d u_i\cdot v_i.
    \end{split}
\end{equation*}

Y dos vectores $u,v\in\R^d$ se dicen ortogonales siempre que $\left\langle u, v \right\rangle=0$.

Sin embargo, en realidad la ortogonalidad va mucho más allá de $\R^d$ y del producto escalar usual. Se puede utilizar cualquier espacio vectorial siempre que esté dotado de un producto escalar. En particular, dado $\Omega\subseteq \R$ pensemos en el espacio de Lebesgue $L^1(\Omega)$, el cual está dotado del producto vectorial
\begin{equation*}
    \begin{split}
        \left\langle\cdot,\cdot\right\rangle:L^1\times L^1 &\longrightarrow \R \\
        (f,g)&\longmapsto \left\langle f,g \right\rangle=\int_{\Omega}f(x)\cdot g(x)dx .
    \end{split}
\end{equation*}

Por ejemplo, si tomamos $\Omega = [0,\pi]$, tenemos que las funciones $\cos(n\theta), \cos(m\theta)$  $(n,m\in\N_0)$ son ortogonales siempre que $n\not= m$, pues 

\begin{equation}
    \label{eq:integralcosenos}
    \int_0^\pi \cos(n\theta)\cos(m\theta) d\theta = 0 \ \ \ (n\not=m).
\end{equation}


Si hacemos el cambio de variable $x = \cos(\theta)$, tenemos que $dx = -\sin(\theta)d\theta=\sin(-\theta)d\theta$, por lo que aplicando que $\sin^2(-\theta)+\cos^2(-\theta)=1$ y que $\cos(-\theta)=\cos(\theta)$, tenemos que la ecuación (\ref{eq:integralcosenos}) se expresa como


\begin{equation}
    \label{eq:integralTn}
    \int_{-1}^1 T_n(x)T_m(x) (1-x^2)^{-\frac 1 2}dx = 0  \ \ \ (n\not=m).
\end{equation}
donde hemos denotado $T_n (x) = \cos(n\theta) = \cos(n\arccos(\theta))$ con $-1\leq x \leq 1$. 

Y tenemos que $T_0=1, T_1=\cos(\theta)=x$, y aplicando identidades trigonométricas se puede deducir que $T_2=2x^2 - 1$, $T_3=4x^3 - 3x$, etc.

Por lo tanto, si definimos en $\mathbb{P}[x]$ el producto escalar

\begin{equation*}
    \begin{split}
        \left\langle\cdot,\cdot\right\rangle:\mathbb{P}[x]\times \mathbb{P}[x] &\longrightarrow \R \\
        (p,q)&\longmapsto \left\langle p,q \right\rangle=\int_{-1}^1 p(x)q(x)\rho(x)dx,
    \end{split}
\end{equation*}
con $\rho(x)=(1-x^2)^{-\frac 1 2}$, tenemos que las funciones (polinomios) $T_n,\ n\in\N_0$ son ortogonales entre sí en el espacio $(\mathbb{P}[x], \left\langle\cdot,\cdot\right\rangle)$ siempre que $n\not=m$.

De acuerdo a este ejemplo, decimos que los polinomios $T_n$ son \textit{ortogonales}, o que la sucesión $\{T_n\}$ es una \textit{Sucesión de Polinomios Ortogonales} con respecto a la \textit{función peso} $\rho(x)=(1-x^2)^{-\frac 1 2}$ en el intervalo $(-1,1)$. Los polinomios $T_n$ recién presentados son los \textit{Polinomios de Tchebichef de tipo I}, y conforman nuestra primera familia de polinomios ortogonales conocida.

\section{Ortogonalidad y función peso}
\label{c1section:ort-peso}

En la sección \ref{c1section:intro} pudimos ampliar el concepto generalmente conocido de ortogonalidad, restringido a espacios como $\R^d$, a otros tipos de espacios. Además, introdujimos la primera familia de polinomios ortogonales: los polinomios de Tchebichef de tipo I. En esta sección daremos definiciones más formales y genéricas sobre la ortogonalidad.

Sea $\alpha$ una función no decreciente y no constante definida en un intervalo $(a,b)$ tal que si $a=-\infty$ entonces $\lim_{x\rightarrow-\infty}\alpha(x)>-\infty$ y si $b=\infty$ entonces $\lim_{x\rightarrow\infty}\alpha(x)<\infty$. Se define el espacio de funciones $L_\alpha^p[a,b]$ como el conjunto de funciones tales que

$$
\int_a^b |f(x)|^p d\alpha(x) < +\infty
$$

En $L_\alpha^2[a,b]$, se define un producto escalar:

\begin{equation}
    \label{eq:prodescLpalpha}
    \begin{split}
        \left\langle\cdot,\cdot\right\rangle:L_\alpha^2[a,b]\times L_\alpha^2[a,b] &\longrightarrow \R \\
        (f,g)&\longmapsto \left\langle f,g \right\rangle=\int_a^b f(x)g(x)d\alpha(x).
    \end{split}
\end{equation}

A partir del espacio $L_\alpha^2[a,b]$ podemos dar una definición de ortogonalidad.

\begin{definicion}[Ortogonalidad]

    Fijada una función $\alpha$ no decreciente y no constante definida en un intervalo $(a,b)$ tal que si $a=-\infty$ entonces $\lim_{x\rightarrow-\infty}\alpha(x)>-\infty$ y si $b=\infty$ entonces $\lim_{x\rightarrow\infty}\alpha(x)<\infty$ y dadas $f,g\in L_\alpha^2[a,b]$, se dice que las funciones $\mathbf f$ \textbf y $\mathbf g$ \textbf{son ortogonales} o que $\mathbf f$ \textbf{es ortogonal a} $\mathbf{g}$ respecto a la distribución $d\alpha$ si
    $$
    \left\langle f,g\right\rangle = 0.
    $$
\end{definicion}

No obstante, en la mayoría de los casos y por simplicidad en lugar de utilizar una función $\alpha$ y su medida inducida, si $\alpha$ es absolutamente continua podemos reescribir (\ref{eq:prodescLpalpha}) como una integral de Lebesgue

\begin{equation}
    \label{eq:deffuncionpeso}
    \left\langle f,g \right\rangle=\int_a^b f(x)g(x)\rho(x)dx,
\end{equation}
siendo $\rho$ una función medible no negativa tal que $0<\int_a^b\rho(x)dx<\infty$ denominada \textbf{función peso}.

\begin{definicion}[Sucesión de Polinomios Ortogonales]
    Dada una función peso $\rho$, si existe una sucesión de polinomios $\{P_n\}$ con $P_n$ de grado $n$ tal que 
    $$
    \left\langle P_n,P_m \right\rangle=\int_a^b P_n(x)P_m(x)\rho(x)dx=0 \ \ \ \ (n\not=m)
    $$
    entonces decimos que $\mathbf{\{P_n\}}$ \textbf{es una Sucesión de Polinomios Ortogonales} (SPO) respecto a la función peso $\rho(x)$ en el intervalo $(a,b)$.  
\end{definicion}

% ─── POLINOMIOS ORTOGONALES CLASICOS ──────────────────────────────────────────

En este apéndice puede consultarse todo lo explicado sobre los polinomios ortogonales clásicos reales: ecuación de Pearson, intervalo de ortogonalidad, función peso, RRTT, ecuación diferencial hipergeométrica, fórmula de Rodrigues\dots Apéndice basado en las deducciones del capítulo \ref{chap:clasicos} y en las tablas de \cite[Ch. XXII]{abramowitz-stegun}.

\begin{table}[h]
    \centering
    \begin{tabular}{cccc}
    \hline
    \textbf{Familia de polinomios} & \textbf{Notación}                  & \textbf{Parámetros}        & \textbf{Intervalo de ortogonalidad} $I$                                               \\ \hline\hline
    Hermite               & $H_n(x)$                  & -                 & $\mathbb R = (-\infty,+\infty)$ \\ \hline
    Laguerre              & $L_n^{(\alpha)}(x)$       & $\alpha>-1$       & $\mathbb R^+_0 = [0,+\infty)$                                                \\ \hline
    Jacobi                & $P_n^{(\alpha,\beta)}(x)$ & $\alpha,\beta>-1$ & $[-1,1]$                                                                     \\ \hline
    \end{tabular}
    \caption{Generalidades sobre los PO clásicos}
    \label{tab:generalidades-clasicos}
    \end{table}

\section{Relación de recurrencia a tres términos}

Los PO clásicos satisfacen la relación de recurrencia
$$
a_{1n} x P_n(x) = a_{2n}P_{n+1}(x) - a_{3n} P_{n}(x) + a_{4n} P_{n-1}(x) \ \ \ \forall n\in\N_0,\; P_{-1}=0, P_0=1, 
$$
donde
\begin{table}[h]
    \begin{tabular}{ccccc}
    \hline
    \textbf{Familia} & \textbf{$a_{1n}$}       & \textbf{$a_{2n}$}                                                                                    & \textbf{$a_{3n}$}                       & $a_{4n}$                                  \\ \hline\hline
    Hermite                        & $2$                     & $1$                                                                                                     & $0$                                     & $2n$                                      \\ \hline
    Laguerre                       & $-1$                    & $n+1$                                                                                                   & $2n+\alpha+1$                           & $n+\alpha$                                \\ \hline
    Jacobi                         & $(2n+\alpha+\beta)_{3}$ &  \begin{tabular}{@{}c@{}}$2(n+1)\times$\\$\times(n+\alpha+\beta+1)\times$ \\ $\times(2n+\alpha+\beta)$\end{tabular}  & \begin{tabular}{@{}c@{}}$(2n+\alpha+\beta+1)\times$ \\ $\times(\alpha^2-\beta^2)$\end{tabular}   &   \begin{tabular}{@{}c@{}}$2(n+\alpha)(n+\beta)\times$ \\ $\times(2n+\alpha+\beta+2)$\end{tabular}\\ \hline
    \end{tabular}
    \caption{Parámetros de la RRTT de las SPO clásicas}
    \label{tab:RRTT-clasicos}
    \end{table}

    Nota: La notación $(y)_n$ denota el símbolo de Pochammer $(y)_n:=y(y+1)\cdots(y+n-1)$.

\section{Ecuación de Pearson}

Las SPO clásicas son ortogonales respecto a una función peso $\rho(x)$ definida en $I$ que es solución de la ecuación de Pearson
$$
[\sigma(x)\rho(x)]'=\tau(x)\rho(x),
$$
donde:


\begin{table}[h]
    \centering
    \begin{tabular}{cccc}
    \hline
    \textbf{Familia} & \textbf{$\sigma(x)$} & \textbf{$\tau(x)$} & \textbf{$\rho(x)$}            \\ \hline\hline
    Hermite          & $1$                  & $-2x$                               & $e^{-x^2}$                    \\ \hline
    Laguerre         & $x$                  & $-x+\alpha+1$                       & $x^{\alpha} e^{-x}$           \\ \hline
    Jacobi           & $1-x^2$              & $-(\alpha+\beta+2)x+(\beta-\alpha)$ & $(1-x)^{\alpha}(1+x)^{\beta}$ \\ \hline
    \end{tabular}
    \caption{Parámetros de la ecuación de Pearson}
    \label{tab:Pearson}
\end{table}

\section{Ecuación diferencial hipergeométrica}

Los polinomios de Hermite, Laguerre y Jacobi son soluciones, para todo $n\in\N_0$, de las siguientes EDOs:

\begin{table}[h]
    \centering
    \extrarowheight = -0.5ex
    \renewcommand{\arraystretch}{1.5}
    \begin{tabular}{cc}
    \hline
    \textbf{Familia} & \textbf{Ecuación diferencial hipergeométrica}                                                                                                 \\ \hline\hline
    Hermite          & $(H_n)'' -2x(H_n)' +2nH_n$                                                                                                           \\[5pt] \hline
    Laguerre         & $x(L_n^{(\alpha)})''+[-x+\alpha+1](L_n^{(\alpha)})' +nL_n^{(\alpha)}$                                                                 \\[5pt] \hline
    Jacobi           & $[1-x^2](P_n^{(\alpha,\beta)})''+[-(\alpha+\beta+2)x+(\beta-\alpha)](P_n^{(\alpha,\beta)})'+n(\alpha+\beta+n+1)P_n^{(\alpha,\beta)}$ \\[5pt] \hline
    \end{tabular}
    \caption{Ecuación diferencial que satisfacen los PO clásicos.}
    \label{tab:EDO}
    \end{table}

\section{Fórmulas de Rodrigues}

Los polinomios de Hermite, Laguerre y Jacobi pueden ser calculados mediante las siguientes fórmulas de Rodrigues:

\begin{table}[h]
    \centering
    \extrarowheight = -0.5ex
    \renewcommand{\arraystretch}{2.25}
    \begin{tabular}{cc}
    \hline
    \textbf{Familia} & \textbf{Fórmula de Rodrigues}                                                                                                   \\ \hline \hline
    Hermite          & $H_n(x)=(-1)^n e^{x^2}\dfrac{d^n}{dx^n}e^{-x^2}$                                                                                \\ \hline
    Laguerre         & $L_n^{(\alpha)}(x)=\dfrac 1 {n!} x^{-\alpha}e^x\dfrac{d^n}{dx^n}(x^{n+\alpha}e^{-x})$                                            \\ \hline
    Jacobi           & $P_n^{(\alpha,\beta)}(x) = \frac{(-1)^n}{2^n n!} (1-x)^{-\alpha}(1+x)^{-\beta}\frac{d^n}{d x^n}\left[(1-x)^{n+\alpha}(1+x)^{n+\beta}\right]$ \\\hline
    \end{tabular}
    \caption{Fórmulas de Rodrigues de los PO clásicos.}
    \label{tab:Rodrigues}
    \end{table}
%
% ─── CONCLUSIONES ───────────────────────────────────────────────────────────────
%

Para finalizar la redacción de esta memoria, pondremos en valor algunas conclusiones extraídas durante el desarrollo de este trabajo. Primero de todo, queremos resaltar la relevancia de los polinomios ortogonales en distintos áreas de las matemáticas. Es bien sabido el interés en aproximar funciones cuya expresión explícita es difícil de computar o incluso, en ocasiones, no es posible obtenerla. Los polinomios son en general una herramienta muy útil en este sentido, pues el conjunto de todos los polinomios $\mathbb P$ es un espacio vectorial, y cualquier problema se facilita considerablemente cuando se cuenta con herramientas de linealidad. En este sentido, los polinomios ortogonales ofrecen bases que convergen rápidamente y ofrecen buenas aproximaciones en ciertos espacios métricos.

También son utilizados en el cálculo de integrales y transformadas, ofreciendo mediante los ceros de las familias de polinomios ortogonales nodos con los cuales los algoritmos de integración numérica proporcionan errores pequeños, incluso con algoritmos poco refinados. Desde el punto de vista simbólico, la propia ortogonalidad puede facilitar el cálculo de integrales complejas.

También pueden ayudar en la resolución de ecuaciones diferenciales, ya que algunas familias de polinomios ortogonales son soluciones polinómicas de ecuaciones diferenciales ordinarias y parciales importantes. Por ejemplo, los polinomios de Hermite y de Legendre son soluciones de la ecuación de Hermite y de Legendre, respectivamente (véase \cite{Hermite-ed}, \cite{Legendre-ed}).

Como vemos, es un área que tiene múltiples conexiones con distintas áreas, además de la destacada en este trabajo, relacionada con la probabilidad y las cadenas y procesos de Markov. Por ello, es una disciplina que está en expansión y que a día de hoy cuenta con un gran equipo de profesionales dedicados a ella. En el ámbito más genérico de los polinomios ortogonales podemos destacar investigadores como Francisco Marcellán, Arno Kuijlaars, Andrei Martínez, Yuan Xu, Kerstin Jordaan, Mourad Ismail o Walter van Assche. Más centrados en el análisis espectral y en relaciones con procesos y cadenas de Markov podemos mencionar a Manuel Domínguez de la Iglesia, Grumbaum F. Alberto o Luis Velázquez.

Con frecuencia se celebran congresos a nivel nacional e internacional en el cual se exponen los últimos avances al respecto. Destacamos el congreso \href{https://w3.ual.es/GruposInv/Tapo/D2PO-2023/D2PO2023.html}{D2PO: Dos Días de Polinomios Ortogonales}\footnote{Web del evento: \url{https://w3.ual.es/GruposInv/Tapo/D2PO-2023/D2PO2023.html}}, cuya última edición se celebró en 2022 en Granada, estando prevista la edición de 2023 en la ciudad de Almería. A nivel internacional, podemos hablar de \href{https://opsfa17.com/}{OPSFA: Orthogonal Polynomials, Special Functions and Applications}\footnote{Web del evento: \url{https://opsfa17.com/}}, cuya edición número 17 tendrá lugar en Granada entre el 24 y el 28 de Junio de 2024.

Actualmente existen varias líneas de investigación abiertas en este campo. Por ejemplo, a pesar de que está muy controlado el comportamiento de los ceros de los PO de una variable, en el caso multidimensional es un problema abierto. De hecho, uno de los problemas relacionados con el ámbito de los ceros de los polinomios ortogonales es el problema número 7 de Smale: la distribución de puntos en la esfera que minimicen la energía logarítmica (véase \cite{Beltran}). También está por estudiar la relación entre cadenas de Markov y polinomios ortogonales en dos o más variables, así como la ortogonalidad múltiple multivariada.

Muchas gracias al lector por su atención, deseo de todo corazón que haya disfrutado leyendo esta memoria como yo redactándola, siempre pensando en usted y en todo el que desee conocer sobre polinomios ortogonales.
